\documentclass[a4paper,12pt]{article}
\usepackage{srcltx}
\usepackage[colorlinks=false,pdfborder=000]{hyperref}
\usepackage{enumerate}
\usepackage{multirow}
\usepackage{titling}
\usepackage[top=1.1in, bottom=1.1in, left=1.2in, right=1.2in]{geometry}
\usepackage[dvips,pdftex]{graphicx,color}
\DeclareGraphicsRule{*}{eps}{*}{}
\usepackage{times}
\usepackage{comment}
\usepackage{subfigure}
\usepackage{amssymb}
%\usepackage[ruled]{algorithm2e}
\usepackage{algorithm}
\usepackage{algorithmic}
\usepackage{subfigure}

%%%%%%%%%%%%%%%%%%%%%%%%%%%%%%%%%%%%%%%%%%%%%%%%%%%%%%%%%%%%%%%%%%%%%%%%%%%%%%%%
\newcommand{\CSE}{\href{http://www.cse.cuhk.edu.hk}{Department of Computer Science and
Engineering}}
\newcommand{\CUHK}{\href{http://www.cuhk.edu.hk}{The Chinese University of Hong Kong}}
\newcommand{\mymail}{\mbox{\textcolor{blue}{\underline{zgxiao@cse.cuhk.edu.hk}}}}
\newcommand{\myname}{\href{http://www.cse.cuhk.edu.hk/~zgxiao}{XIAO Zigang}}
\newcommand{\header}[1]{\noindent {\bf \\#1\\}}
\renewcommand{\algorithmiccomment}[1]{//#1}
%%%%%%%%%%%%%%%%%%%%%%%%%%%%%%%%%%%%%%%%%%%%%%%%%%%%%%%%%%%%%%%%%%%%%%%%%%%%%%%%
\pretitle{\begin{center}\bf \LARGE} \posttitle{\par\end{center}}
\preauthor{\begin{center}
            \small \lineskip 0.5em%
            \begin{tabular}[t]{c}}
\postauthor{\end{tabular}\par\end{center}}
\predate{\begin{center}\small} \postdate{\par\end{center}}
%%%%%%%%%%%%%%%%%%%%%%%%%%%%%%%%%%%%%%%%%%%%%%%%%%%%%%%%%%%%%%%%%%%%%%%%%%%%%%%%
\title{CEG5270 Assignment 2 (2009 Spring)}
\author{\myname\\\mymail\\\CSE\\\CUHK}
\date{\today}

%%%%%%%%%%%%%%%%%%%%%%%%%%%%%%%%%%%%%%%%%%%%%%%%%%%%%%%%%%%%%%%%%%%%%%%%%%%%%%%%
\begin{document}
\maketitle
\begin{enumerate}
%%%%%%%%%%%%%%%%%%%%%%%%%%%%%%%%%%%%%%%%%%%%%%%%%%%%%%%%%%%%%%%%%%%%%%%%%%%%%%%%
\item
First it is easy to know that for any pin pair, where one is in the top side and one is in the bottom side,
they can use different vertical layers for routing and will not cause conflict, if there are two vertical layers.
Then for the track assignment, since each net corresponds to a horizontal segment, 
it turns out to be problem which finds the chromatic number of a segment graph.
It is proved in assignment I that greedy algorithm similar to LEA can produce the optimal result.
Since we know that the lower bound of the channel width is the max of:
	\begin{enumerate}
	\item\label{lb1}
	length of the longest path in VCG
	\item\label{lb2}
	channel density
	\end{enumerate}
which also equals to the maximum clique in the segment graph.
And we know that maximum clique size equals to chromatic number in the segment graph.
Then we proved that LEA gives solution of minimum channel width.
%%%%%%%%%%%%%%%%%%%%%%%%%%%%%%%%%%%%%%%%%%%%%%%%%%%%%%%%%%%%%%%%%%%%%%%%%%%%%%%%
\item
At most $K=4$ times of gain update is needed due to the change of a net.

Since only critical nets will cause gain update, 
we only need to consider those nets which are critical before or after the move.
We observe that once two cells on a net are moved to X and another two cells are moved to Y,
they are locked and it is impossible to make the net to be critical anymore.
Hence it is the bound of updates. The worst case is at most 4, 
if we assume every move bring changes before the net reaches the state mentioned above.
For example, originally $X=\{A,B\},Y=\{C,D\}$, by moving them in the order $A,D,B,C$ will produce 4 gain updates, 
but no further update anymore.
%%%%%%%%%%%%%%%%%%%%%%%%%%%%%%%%%%%%%%%%%%%%%%%%%%%%%%%%%%%%%%%%%%%%%%%%%%%%%%%%
\item
Without loss of generality, we assume the routing direction is anti-clockwise.
For a 2-pin net $i$, if there is net, say $j$,inside the bounding segment formed by $i$'s pins, 
then $i$ will conflict with $j$ if and only if NOT all of $j$'s is inside the bounding segment,
e.g. for net $b$, only $c_1$ is inside the bounding segment hence $c$ is conflict with $b$.
But $a$ is not conflit with $b$ since all the pins of $a$,i.e. $a_1,a_2$ is inside $b_2$ to $b_3$.
For 
\begin{algorithm}[htbp!]
\caption{Find Largest Routable Net Numer}
\begin{algorithmic}[1]
\REQUIRE A list of net pins $plist$, the starting terminal of each net is known.\\
\STATE // construct of the cycle graph can be done in $O(n)$ time, where $n$ is the number of pins.
\RETURN 
\end{algorithmic}
\end{algorithm}
Suppose the pin number of net is bounded by a constant $P$, and there are totally $n$ nets.
The runing time of the algorithm would be $(Pn)^2=O(n^2)$.Thus the whole
%%%%%%%%%%%%%%%%%%%%%%%%%%%%%%%%%%%%%%%%%%%%%%%%%%%%%%%%%%%%%%%%%%%%%%%%%%%%%%%%
\item
There are only three possible cases of exchange.They are:
\begin{itemize}
\item
Exchange a corner node with another corner node. This gives a gain of 0.
\item
Exchange a corner node with a non-corner node.This gives a gain -1.
\item
Exchange a non-corner node with another non-corner node. This gives a gain -2.
\end{itemize}
We notice that none of them produces a positive gain, 
hence in the process of finding $k$ s.t. $G=g_1+g_2+...+g_k$, the maximized value 0 should be reached when $k=n$.
The algorithm stops here. It is a local minimum of K-L.
\end{enumerate}
\end{document}
% \begin{algorithm}[htbp!]
% \caption{Find Largest Routable Net Numer}
% \begin{algorithmic}[1]
% \REQUIRE A list of net pins $plist$, the starting terminal of each net is known.\\
% \STATE queue $\leftarrow$ plist // put all elements of list into a queue
% \WHILE { queue not empty }
% 	\STATE $p \leftarrow $ queue.front()
% 	\STATE $x \leftarrow $ pinOfNet(p) // check pin in which net
% 	\IF {$p$ == $x$.startingTerminal } 
% 		\STATE stack.push($p$) 
% 		\STATE mark[$p$] $\leftarrow$ TRUE // mark starting terminal in stack
% 		\STATE queue.remove($p$)
% 	\ELSIF { mark[$p$] == TRUE }
% 		\STATE // $p$ is not starting terminal, and starting terminal is in stack 
% 		\STATE $j \leftarrow $ index of stack.top()
% 		\IF { stack[$j$] == $x$.staringTerminal }
% 			\STATE // do not intersect with any nets
% 			\STATE stack.pop()
% 		\ELSE 
% 			\STATE mark al the net between this net as conflict
% 			\WHILE { stack[$j$] $\neq$ $x$.staringTerminal }
% 				\STATE $y \leftarrow $ pinInNet(stack[$j$])
% 				\STATE conflictList[$x$].add(stack[$j$])
% 				\STATE $j\leftarrow j-1$
% 			\ENDWHILE
% 		\ENDIF
% 		\STATE queue.remove($p$)
% 	\ELSE % ignore this net,put it into queue again
% 		\STATE queue.remove($p$)
% 		\STATE queue.pushback($p$)
% 	\ENDIF
% \ENDWHILE
% \STATE // now we have the conflict information, choose those net conflict with least nets
% \RETURN 
% \end{algorithmic}
% \end{algorithm}
