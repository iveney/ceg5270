\documentclass[a4paper,12pt]{article}
\usepackage{srcltx}
\usepackage[colorlinks=false,pdfborder=000]{hyperref}
\usepackage{enumerate}
\usepackage{multirow}
\usepackage{titling}
\usepackage[top=1.2in, bottom=1.2in, left=1.2in, right=1.2in]{geometry}
\usepackage[dvips,pdftex]{graphicx,color}
\DeclareGraphicsRule{*}{eps}{*}{}
\usepackage{times}
\usepackage{comment}
\usepackage{subfigure}
\usepackage{amssymb}
\usepackage{algorithm}
\usepackage{algorithmic}
%%%%%%%%%%%%%%%%%%%%%%%%%%%%%%%%%%%%%%%%%%%%%%%%%%%%%%%%%%%%%%%%%%%%%%%%%%%%%%%%
\newcommand{\CSE}{\href{http://www.cse.cuhk.edu.hk}{Department of Computer Science and
Engineering}}
\newcommand{\CUHK}{\href{http://www.cuhk.edu.hk}{The Chinese University of Hong Kong}}
\newcommand{\mymail}{\mbox{\textcolor{blue}{\underline{zgxiao@cse.cuhk.edu.hk}}}}
\newcommand{\myname}{\href{http://www.cse.cuhk.edu.hk/~zgxiao}{XIAO Zigang}}
\newcommand{\header}[1]{\noindent {\bf \\#1\\}}
%%%%%%%%%%%%%%%%%%%%%%%%%%%%%%%%%%%%%%%%%%%%%%%%%%%%%%%%%%%%%%%%%%%%%%%%%%%%%%%%
\pretitle{\begin{center}\bf \LARGE} \posttitle{\par\end{center}}
\preauthor{\begin{center}
            \small \lineskip 0.5em%
            \begin{tabular}[t]{c}}
\postauthor{\end{tabular}\par\end{center}}
\predate{\begin{center}\small} \postdate{\par\end{center}}
%%%%%%%%%%%%%%%%%%%%%%%%%%%%%%%%%%%%%%%%%%%%%%%%%%%%%%%%%%%%%%%%%%%%%%%%%%%%%%%%
\title{CEG5270 Assignment 1 (2009 Spring)}
\author{\myname\\\mymail\\\CSE\\\CUHK}
\date{\today}

%%%%%%%%%%%%%%%%%%%%%%%%%%%%%%%%%%%%%%%%%%%%%%%%%%%%%%%%%%%%%%%%%%%%%%%%%%%%%%%%
\begin{document}
\maketitle
\begin{enumerate}
\item

\begin{algorithm}[htbp!]
\begin{algorithmic}[1]
\STATE Sort the intervals according to the start and end point and save it in a list $L$.(hence there are $2n$ elements)
\STATE Mark all intervals as not colored.
\STATE freelist $\leftarrow \{1,2,...,n\}$
\STATE usedlist $\leftarrow \varnothing$
\STATE $k \leftarrow 0$ 
\FOR {$i$ in [1..2n]}
	\IF {$L[i]$ is start point of interval $x$}
		\STATE $c \leftarrow $ head of freelist
		\STATE set color of $x$ as $c$
		\STATE remove $c$ from freelist
		\STATE insert $c$ into usedlist
		\IF {$c > k$}
			\STATE $k \leftarrow c$
		\ENDIF
	\ELSE 
		\STATE // $L[i]$ is end point of interval $x$
		\STATE $c \leftarrow $the color value of $x$
		\STATE remove $c$ from usedlist
		\STATE insert $c$ to the head of freelist
	\ENDIF
\ENDFOR
\RETURN  $k$.
\end{algorithmic}
\end{algorithm}
Use greedy algorithm to solve this problem.

Proof: Let $x$ be the vertex which is colored to $k$. Since $x$ is not colored to a smaller value,
it means $x$'s start point $a$ intersects with the intervals which are colored from $1$ to $k-1$, 
i.e. those intervals includes $a$. Then these intervals forms a $k-$clique. 
Hence $\omega(G) \geq k \geq \chi(G)$.But we know that $\chi(G) \geq \omega(G)$. 
Hence $\omega(G) = \chi(G) = k$.

Complexity: Sorting $2n$ elements takes $O(n\log n)$. 
And it takes $O(n)$ for the for loop in the algorithm, only $O(1)$ for the inserting and removing operation of queues.
Hence the total complexity is dominated by the sorting time $O(n\log n)$.
%%%%%%%%%%%%%%%%%%%%%%%%%%%%%%%%%%%%%%%%%%%%%%%%%%%%%%%%%%%%%%%%%%%%%%%%%%%%%%%%
\item
Let $H[k][i][j]$ denotes the number of at most K multi-bend routes from s to t where the last step is horizontal move.
Let $V[k][i][j]$ denotes the number of at most K multi-bend routes from s to t where the last step is vertical move.
And they are defined as follows:
\begin{equation}
H[k][i][j] = \left\{ \begin{array}{ll}
			0	&	\textrm{if $j=0$ or $i\neq0,j\neq0,k=0$}\\
			1	&	\textrm{if $j\neq 0,i=0 $}\\
			H[k][i][j-1] + V[k-1][i][j-1]&	\textrm{otherwise}\\
                    \end{array}\right.
\end{equation}
\begin{equation}
V[k][i][j] = \left\{ \begin{array}{ll}
			0	&	\textrm{if $i=0$ or $i\neq0,j\neq0,k=0$}\\
			1	&	\textrm{if $i\neq 0,j=0 $}\\
			H[k-1][i-1][j] + V[k][i-1][j] &	\textrm{otherwise}\\
                    \end{array}\right.
\end{equation}
Without loss of generality, let $(0,0)$ and $(x,y)$ be the coordinate of $s,t$, 
then $H[k][x][y]+V[k][x][y]$ is the number of multi-bend routes from s to t.
%%%%%%%%%%%%%%%%%%%%%%%%%%%%%%%%%%%%%%%%%%%%%%%%%%%%%%%%%%%%%%%%%%%%%%%%%%%%%%%%
\item
\begin{enumerate}[(i)]
\item
\item
\end{enumerate}
%%%%%%%%%%%%%%%%%%%%%%%%%%%%%%%%%%%%%%%%%%%%%%%%%%%%%%%%%%%%%%%%%%%%%%%%%%%%%%%%
\item
\begin{enumerate}[(i)]
\item
$C:$ $q[j]<i$ and no edge between $q[j]$ and $Largest$\\
$D:$ $Largest=q[j]$

Explanation: Condition $C$ means that if for some number $x=q[j]$<i, 
then they have the relationship like $(...x...i...),(...x...i...)$ in the sequence pair.
Also it checks if it will be a transitive edge.
Statement $D$ helps to update the $Largest$ so as to keep track of the transitive edge information.
\item
Change line 3 to:\\
\textit{For(j=k+1 up to n) do}

Replace $G_v$ for every occurence of $G_h$ in the algorithm.
\item
Since every pair of modules has a relation, the total number of edges would be $C^2_n$ with transitive edges.
In the worst case of horizontal constraint graph, all the modules are placed one next to another.
But if transitive edges are removed there will be at most $(n-1)$ edges. 
This result holds in vertical constraint graph.
Hence the number of edges in the constraint graph can be reduce to $O(n)$.
\end{enumerate}
%%%%%%%%%%%%%%%%%%%%%%%%%%%%%%%%%%%%%%%%%%%%%%%%%%%%%%%%%%%%%%%%%%%%%%%%%%%%%%%%
\item

\end{enumerate}
\end{document}
